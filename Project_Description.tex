\documentclass{article}
\usepackage[utf8]{inputenc}
\usepackage{amsmath,amssymb,amsfonts, mathtools}
\usepackage{enumitem}                   
\usepackage{listings}                   % insert code
\usepackage{graphicx}                   % Insert images
%\graphicspath{ {images/} }
\usepackage[letterpaper, margin=1.5cm]{geometry}
\usepackage[singlespacing]{setspace}

\begin{document}
\noindent
\large \textbf{ECSE 414 Intro. to Telecom. Networks} \hfill \textbf{ADD YOURS} \\
\large \textbf{Deliverable \#1 - Description} \hfill \textbf{Camilo Garcia La Rotta} \textbf{\#260657037} \\

\section*{Methodology}

\section*{Testing}
The goal of the project is to achieve a minimalist implementation of onion routing and perform analysis of its performance via simulation. As such, testing of this project should be divided into two sections:
\begin{enumerate}
    \item \textbf{Verify and validate the onion routing implementation:} We will follow a Test Driven Development (TDD) paradigm. Writing test cases first, then deriving the unit tests through equivalence partitioning and finally writing the necessary code for all the tests to pass. Throughout this iterative process we will rely on Continuous Integration (CI) to efficiently manage all the source code generated by each sub-team. The concept of pair programming will be leveraged in order to equally subdivide the tasks among the group and to ensure regular code reviews are performed on the code base. With respect to the application testing, we shall focus on three main aspects of onion routing: \underline{bidirectional communication}, \underline{encryption of data through nodes} and \underline{randomized path selection}. Each of these characteristics will be unit tested with virtual machines and after satisfying our requirements they will be undergo integration testing. In order to efficiently replicate the development and testing environment across the machines of each team member, Vagrant will be used to build and maintain the portable virtual development environments.
    \item \textbf{Analyze the performance of the application through simulation:}
    
% As for the second section of testinng (performance) we gotta find some tools for network testing (i.e. nmap, wireshark) and talk about 
% how we will benchmark those tests ina very high level manner

\end{enumerate}


\section*{BlaBlaBla}


\end{document}
