\documentclass[10pt]{article}
\usepackage[utf8]{inputenc}
\usepackage{amsmath,amssymb,amsfonts, mathtools}
\usepackage{enumitem}
\usepackage{listings}                   % insert code
\usepackage{graphicx}                   % Insert images
%\graphicspath{ {images/} }
\usepackage[letterpaper, margin=1.4cm]{geometry}
\usepackage[singlespacing]{setspace}

\begin{document}
\noindent
\large \textbf{ECSE414 - Intro to Telecom Networks\\Deliverable 1 - Project Description: Onion Routing: Group 10}\\
Authors: C. Garcia La Rotta, D. Liu, E. Zhao, H. Wiltzer, J. Virdee, S.D. Mavroidakos, S. Mashaal\\
September 29, 2017

\section*{Project Description}

The goal of this project is to implement and design an Onion Routing protocol,
which ideally allows for users to communicate completely anonymously over the
internet. Communication with Onion Routing involves peers on a network through which data is transmitted with multiple layers of encryption. These encryption layers get decrypted as
the message is passed throughout the intermediate peers in the path.
However, as opposed to proxy servers for example,
the layered encryption ensures that no nodes along the
message path are aware of who the sender was, so it is impossible to determine
if a particular sender ever actually sent a message. Our Onion Routing
system will support encryption on bidirectional communication to achieve this
anonymity.\\

\noindent Performance of the system will also be assessed via simulation. A network topology which mimics characteristics of the internet will be used. Onion routing will be evaluated by comparing the lengths of routes taken by a packet using onion routing to one using shortest-path routing. Additional delay will also be included to calculate the total price of anonymity.


\section*{Methodology}

The Onion Routing Protocol will be implemented with several independent routers (nodes)
communicating via TCP packets.  For all $N$ nodes, each node $n_i$ has
an asymmetric cryptographic key pair $(pub_{i},priv_{i})$ and an IP address $a_i$.  Additionally, a special
server called the Directory Node contains the information $\{(a_i, pub_i) \forall i \in [\![1..N]\!]\}$.\\

\noindent Say a sender $\mathcal{S}$ wishes to send a request (and receive a response)
from some destination $\mathcal{D}$.  Firstly, $\mathcal{S}$ connects to the Directory Node
and requests a path $\mathbf{P}$ through the network. The Directory Node then generates
$\mathbf{P}$, a random and ordered subset of the onion-net's $N$ nodes where $|\mathbf{P}| \leq N$.  For each node
$n_j \in \mathbf{P}$, the Directory Node negotiates using the corresponding
$pub_j$ and its own private key, thereby producing a symmetric key $sym_j$.  The Directory
Node then constructs a packet containing $\{(a_j, sym_j) \forall j \in [\![1..|\mathbf{P}|]\!]\}$ and
encrypts it with the public key corresponding to $\mathcal{S}$, and sends the cipher to $\mathcal{S}$.\\

\noindent Now, $\mathcal{S}$ decrypts the cipher with his private key. He creates the
request he wishes to send to $\mathcal{D}$, and encrypts with each $sym_j$
successively, but in reverse order (so the first $sym_j$ he encrypts with is $sym_{|\mathbf{P}|}$, which corresponds
to the last node in the path). Before encrypting with each $sym_j$, a header
indicating which node to send to next is prepended so that each node can
determine where to send the packet after decrypting. $\mathcal{S}$ proceeds to
send this many-layered ``onion'' of ciphers to the first node in the path.\\

\noindent Each intermediate node $n_j$ behaves according to the exact same protocol. Upon receiving
a packet, a node $n_j$ decrypts the packet with its symmetric key to expose $a_{j+1}$, the
address of the next node $n_{j+1}$ in the path.  After sending the packet to $n_{j+1}$, $n_j$
waits for a response packet.  When a response packet is received, $n_j$ encrypts the response with
$sym_j$ and sends the packet to $n_{j-1}$.  When $\mathcal{S}$ receives the response, he decrypts
with each symmetric key successively (now in forward order), and can then read the response.\\

\noindent All nodes will be implemented as TCP servers programmed with the Python programming
language using the Flask framework to handle concurrent requests.  The Vagrant tool will be
used to automate the creation of virtual machines which will simulate the nodes.  Additionally,
to save processing power, the Directory Node will run locally on the sender's machine for the
purposes of the simulation.\\

\noindent Furthermore, the NTRU cryptoscheme will be considered and investigated to
provide key exchange for communicating the $s_i$ keys, and can maintain
security against quantum opponents.


\section*{Testing}
The goal of the project is to achieve a minimalist implementation of onion
routing and perform analysis of its performance via simulation. As such,
testing of this project should be divided into two sections:

\noindent \textbf{Verify and validate the onion routing implementation:} We
will follow a Test Driven Development (TDD) paradigm. Writing test
cases first, then deriving the unit tests through equivalence
partitioning and finally writing the necessary code for all the tests
to pass. Throughout this iterative process each sub-team will rely on GitHub for the source code management tool. We will leverage GitHub's integrated bug tracking and ticketing system in
order to equally subdivide the tasks among the group and to ensure
regular code reviews are performed on the code base. Integration of TravisCI to our GitHub repository will enable us to perform Continuous Integration (CI) to ensure regression testing every time new functionalities or bug fixes are pushed to the master branch. With respect to
the application testing, we shall focus on three main aspects of onion
routing: \textit{bidirectional communication}, \textit{encryption
of data through nodes} and \textit{randomized path selection}. Each
of these characteristics will be unit tested with virtual machines and
after satisfying our requirements they will be undergo integration
testing. In order to efficiently replicate the development and testing
environment across the machines of each team member, Vagrant will be
used to build and maintain the portable virtual development
environments. \\

\noindent \textbf{Analyze the performance of the application through
simulation:} There will be two tools that will be used, one is Nmap
which provides a broader overview of the network and Wireshark to
examine the packets transferred between nodes. We will be using Nmap
for testing enumeration of network components and available application
layer protocols at end hosts. It will also be used to identify any
vulnerabilities in the network and to ensure the ports are connected to
the correct hosts. Wireshark is used to test encryption requirements,
provide an analysis of the chosen network and ensure contiguous nodes
know the adjacent source destination and not the global ones. For
further testing, we may use the mentioned tools such as Brite or Orbis
to generate network topology to simulate characteristics in the
internet.


\section*{Work Division}
The work for this project will be divided into two main subteams, one of which
will be the development subteam, and the other will be the testing subteam. As
it stands, Camilo, Harley, Spiros-Daniel, and Stuart will make up the
development subteam, and Dennis, Edward, and Jastaj will make up the testing
subteam. It has been decided that each team member will document his own work,
and Harley has volunteered to manage and compile the documentation that
everyone writes. Furthermore, due to the difficulty of testing this system,
Spiros-Daniel has volunteered to help out the testing team, so that the testing
team can have assistance with testing out the more intricate parts of the
system.\\\\
The development team will be in charge of developing the code and the software
and network architectures. Furthermore, to follow a TDD paradigm, the
development team will be in charge of writing unit tests for the features that
they will be implementing.\\
The testing team will be in charge of conducting thorough integration and
system tests that should reveal the behavior of the system under ordinary and
stressful circumstances. The testing team will then report bugs or improvement
ideas to the development team as they find them. Furthermore, the testing team
will be in charge of setting up and managing the Continuous Integration of our
software.

\end{document}